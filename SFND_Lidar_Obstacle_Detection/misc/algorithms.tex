\documentclass[12pt]{article}

\title{Algorithms for LIDAR Obstacle Detection}
\author{Jun Zhu}
\date{\today}


\usepackage[ruled,vlined]{algorithm2e}

\usepackage{geometry}
\geometry{
	a4paper,
	left=20mm,
	right=20mm,
	top=20mm,
	bottom=20mm
}

\begin{document}
\maketitle

%%%%%%%%%%%%%%%%%%%%%%%%%%%%%%%%%%%%%%%%%%%%%%%%%%%%%%%%%%%%%%%%%%%%%%
%%%%%%%%%%%%%%%%%%%%%%%%%%%%%%%%%%%%%%%%%%%%%%%%%%%%%%%%%%%%%%%%%%%%%%
%%%%%%%%%%%%%%%%%%%%%%%%%%%%%%%%%%%%%%%%%%%%%%%%%%%%%%%%%%%%%%%%%%%%%%

\begin{algorithm}[H]
\DontPrintSemicolon
\SetAlgoLined
\SetKwInOut{Input}{Input}
\SetKwInOut{Output}{Output}
\Input{$D$ - Input point cloud. \\
       $k$ - Maximum number of iterations allowed. \\
	     $tol$ - Threshold value to determine whether a point belongs to a plane.
}

\Output{Points belonging to a plane.}

\BlankLine

Initialize a list to store the best inliners found so far, $bestI$\;
$iterations \leftarrow 0$\;
\While{ $iterations < k$ } {
	  Initialize a list to store inliners for the current iteration,  $I$\;
    Randomly select 3 points from $D$ and add them to $I$\;
    Calculate the coefficients of the plane function\;
		\For{ \normalfont{every point in} $D$ } {
				Calculate its distance to the plane, $dist$\;
				\If{ $dist \leq tol$ } {
						Add this point to $I$\;
				}
		}
		\If{ \normalfont{size of} $I > $ \normalfont{size of} $bestI$ } {
				$bestI \leftarrow I$\;
		}
}
\caption{RANSAC for 3D point cloud plane segmentation}
\end{algorithm} 

%%%%%%%%%%%%%%%%%%%%%%%%%%%%%%%%%%%%%%%%%%%%%%%%%%%%%%%%%%%%%%%%%%%%%%
%%%%%%%%%%%%%%%%%%%%%%%%%%%%%%%%%%%%%%%%%%%%%%%%%%%%%%%%%%%%%%%%%%%%%%
%%%%%%%%%%%%%%%%%%%%%%%%%%%%%%%%%%%%%%%%%%%%%%%%%%%%%%%%%%%%%%%%%%%%%%

\newpage

\begin{algorithm}[H]
\DontPrintSemicolon
\SetAlgoLined
\SetKwInOut{Input}{Input}
\SetKwInOut{Output}{Output}
\Input{$p$ - Reference point. \\
       $root$ - Root node of the k-d tree. \\
	     $tol$ - Maximum allowed distance for a neighboring point.
}

\Output{A list of neighboring points.}

\BlankLine

\Begin{
		Initialize a list to store the neighboring points, $I$\;
		TreeSearch($p$, $root$, 0, $tol$, $I$)\;
		return $I$\;
}

\BlankLine

\SetKwFunction{TreeSearch}{TreeSearch}
\SetKwProg{Fn}{Function}{:}{}
\Fn{\TreeSearch{$p$, $node$, $depth$, $tol$, $I$}}{
		\If{$node == null$} {
				return\;
		}
		Compute the distance between $p$ and $node$, $dist$\;
		\If{$dist <= tol$} {
				Add $node$ to $I$\;	
		}
		
		\If{it is worth to travel left} {
				TreeSearch($p$, left child of $node$, depth+1, $tol$, $I$);	
		}
		\If{it is worth to travel right} {
				TreeSearch($p$, right child of $node$, depth+1, $tol$, $I$);	
		}
}

\caption{Find neighbors using k-d tree}
\end{algorithm} 

\end{document}